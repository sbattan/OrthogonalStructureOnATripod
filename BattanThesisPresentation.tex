\documentclass{beamer}[10]
\usepackage[danish]{babel}
\usepackage{beamerthemesplit}
\usepackage{hyperref}
\usepackage{amsmath, amsthm, amssymb, amsthm, tikz, pgfplots}
\pgfplotsset{compat=1.14}

\definecolor{sierra}{RGB}{18,71,52}
\setbeamercovered{transparent}
\mode<presentation>
\usetheme{PaloAlto}
\setbeamertemplate{footline}[page number]
\setbeamerfont{footline}{size=\tiny}
\setbeamercolor{footline}{fg=white}
\setbeamertemplate{navigation symbols}{}

\usecolortheme[named=sierra]{structure}
\useinnertheme{circles}
\usefonttheme[onlymath]{serif}
\setbeamercovered{transparent}
\setbeamertemplate{blocks}[rounded][shadow=true]

\newtheorem{defn}{Definition}[section]
\newtheorem{thm}{Theorem}[section]

\logo{\includegraphics[width=1.5cm]{Uoseal.png}}
\title{Orthogonal Structure \\ on a Tripod}
\author{Sierra Nicole Battan}
\institute{Robert D. Clark Honors College \\ Department of Mathematics \\ University of Oregon}
\date{4 March 2019}

\begin{document}

\pgfkeys{/pgfplots/scale/.style={
  x post scale=#1,
  y post scale=#1,
  z post scale=#1}
}

\pgfkeys{/pgfplots/axis labels at tip/.style={
    xlabel style={at={(current axis.right of origin)}, xshift=1.5ex, anchor=center},
    ylabel style={at={(current axis.above origin)}, yshift=1.5ex, anchor=center}}
}





\frame{\titlepage}





\section{Background}																			% Background %


\frame{\frametitle{What Is Going On?}
	\begin{itemize}
		\item Intersection of multivariable calculus, linear algebra, and numerical mathematics.
		\item Extension of the forefront of numerical research that defines a structure with which mathematicians can manipulate monster polynomials.
		\item Presentation of four increasingly efficient algorithms that construct an orthogonal basis that accurately represents the structure of polynomials on any 3D tripod.
	\end{itemize}
}
\frame{\frametitle{What Is Orthogonality?}
	\ \ \ \ \ \ \ \ \begin{tikzpicture}	
		\begin{axis}
			[scale=0.5, axis y line=center, axis x line=center, xmin=-1, xmax=1, ymin=-1, ymax=1, xticklabels={,}, yticklabels={,}]
		\end{axis}
	\end{tikzpicture}
	\begin{tikzpicture}
		\begin{axis}
			[scale=0.9, view={60}{30}, axis y line=center, axis x line=center, axis z line=center, xmin=-1, xmax=1, ymin=-1, ymax=1, zmin=-1, zmax=1, xticklabels={,,}, yticklabels={,,}, zticklabels={,,}]
	\end{axis}
	\end{tikzpicture}
	\begin{itemize}
		\item Orthogonality between two functions: $\langle f,g\rangle = 0$.
		\\ \ \ \ - \ Orthogonal means $90^\circ$ and $\cos(90)=0$.
		\\ \ \ \ - \ Euclidean: $\cos(\theta) = \frac{\langle \mathbf{x},\mathbf{y}\rangle}{||\mathbf{x}|| \ ||\mathbf{y}||}$. %, where $\theta$ is the angle between two vectors $x$ and $y$. 
		\item Higher dimensions use sums of functions $f = \sum a_n P_n (x)$.
		\item Inner product $\langle f, g\rangle$ must be commutative, linear in each component, and positive definite.
	\end{itemize}
}
\frame{\frametitle{What Are Iterative And Recursive Algorithms?}
	\begin{itemize}
		\item Solving a larger problem in a series of smaller steps that may or may not depend on the previous step.
		\item The solution to our current execution of the algorithm(s) utilizes results from the execution of previous algorithm(s).
		\item Specifically, three-term recursion: \\ $F_n = F_{n-1} + F_{n-2}$ with initial values $F_0 = 0$ and $F_1 = 1$.
		\item Minimizes dependencies and redundant computations \\ $\Longrightarrow$ improves algorithmic efficiency.
	\end{itemize}
}
\frame{\frametitle{Research Goals \& Applications}
	\begin{itemize}
		\item Exploit orthogonality, iteration, recursion, and computerized mathematics to improve algorithmic efficiency, which is crucial in the world of ``big data."
		\item Approximate polynomials on a 3D tripod by using orthogonal infinite sums.
		\item Fourier orthogonal series and signal readings: $$f = \sum_{i=1}^\infty a_i P_i + \sum_{i=1}^\infty b_i Q_i + \sum_{i=1}^\infty c_i R_i + d_0.$$
		\item Expand the current study of all orthogonal structures.
	\end{itemize}
}





\section{Foundations}																			% FOUNDATIONS %


\frame{\frametitle{Desired Inner Product Space}
	\begin{align*}
		\langle f, g\rangle &= \int_0^1 f(x,0,0)g(x,0,0) dx \\
					&+  \int_0^1 f(0,y,0)g(0,y,0) dy \\
					&+ \int_0^1 f(0,0,z)g(0,0,z) dz
	\end{align*}
	
	\vspace*{2mm}
	Commutative, linear in each component, and positive definite: \\ \ \ \ \ \ $\langle f, f\rangle \geq 0$ where $\langle f, f\rangle=0$ if and only if $f$ itself equals $0$.
}
\frame{\frametitle{Domain Restriction With Polynomial Ideals}
	General $\mathbb{R}_n [x, y, z]$ bases:
		$$\begin{array}{lllllllllll}
			n=0: & 1 \\
			n=1: & x & y & z \\
			n=2: & x^2 & xy & xz & y^2 & yz & z^2 \\
			n=3: & x^3 & x^2y & xy^2 & x^2z & xz^2 & y^3 & y^2z & yz^2 & z^3 & xyz.
		\end{array}$$
	Polynomial Ideals: $f \in \langle p_1, \ldots, p_n\rangle \Leftrightarrow \exists q_i : f = \sum_{i=1}^n\{ p_i\times q_i \}.$
	
	\vspace*{2mm}
	Our inner product space cannot contain $\langle xy, xz, yz \rangle$.
}
\frame{\frametitle{Inner Product Space's Domain Dimension}
	Restricted $\mathbb{R} [x, y, z] / \langle xy, xz, yz \rangle$ bases:
		$$\begin{array}{llll}
			n=0: & 1 \\
			n=1: & x & y & z \\
			n=2: & x^2 & y^2 & z^2 \\
			n=3: & x^3 & y^3 & z^3.
		\end{array}$$
	\begin{thm} Our inner product space's restrictive domain has dimension:
		$$\text{When }n=0, dim(\mathbb{R}_n [x, y, z] / \langle xy, xz, yz\rangle) = 1.$$
		$$\forall n\in \mathbb{N} : n > 0, dim(\mathbb{R}_n [x, y, z] / \langle xy, xz, yz\rangle) = 3.$$
	\end{thm}
}





\section{Algorithms}																		% ALGORITHMS %


\frame{\frametitle{Na\"ive Algorithm}
	$V_n = \mathbb{R}_n [x, y, z] / \langle xy, xz, yz\rangle$ is the $n^{th}$-degree inner product space of the restricted tripod.
	
	\vspace*{2mm}
	Monic: $P_n = x^n + \langle$ lower-degree combination of$x,y,z\rangle$.
	
	\vspace*{2mm}Construct the monic polynomial basis $B_n = \{P_n, Q_n, R_n\}$:
		$$P_n(x,y,z) = x^n + \sum_{i=1}^{n-1}a_i^Px^i + \sum_{i=1}^{n-1}b_i^Py^i + \sum_{i=1}^{n-1}c_i^Pz^i + d_0^P.$$
	Since each basic element is monic, we can determine them by $\langle P_n, x^j\rangle = \langle P_n,y^j\rangle = \langle P_n, z^j\rangle = 0.$
		
	\vspace*{2mm}
	$\ldots$
}
\frame{\frametitle{Na\"ve Algorithm Complexity}
	Three $(3n-2)\times(3n-2)$ matrices:
	
	\vspace*{2mm}
	\ \ \ \ \ \
	\begin{tikzpicture}\draw[color=gray,step=.1cm] (0.5,0.5) grid (3.3,3.3);\end{tikzpicture}
	\begin{tikzpicture}\draw[color=gray,step=.1cm] (0.5,0.5) grid (3.3,3.3);\end{tikzpicture}
	\begin{tikzpicture}\draw[color=gray,step=.1cm] (0.5,0.5) grid (3.3,3.3);\end{tikzpicture} .
	
	\vspace*{2mm}
	Given $B_n = \{P_n, Q_n, R_n\}$ we can use elementary linear algebra techniques to construct $OB_n = \{\hat{P}_n, \hat{Q}_n, \hat{R}_n\}$. 
}
\frame{\frametitle{Basic Algorithm}
	Introduce iteration to similarly construct $B_n = \{P_n, Q_n, R_n\}$:
	\begin{align*}
	P_n(x, y, z) = x^n + \sum_{i=1}^{n-1} a_i^P&P_i(x,y,z) + \sum_{i=1}^{n-1} b_i^PQ_i(x,y,z) \\
				&+ \sum_{i=1}^{n-1} c_i^PR_i(x,y,z) +d_0^P.
	\end{align*}
	Determine each element for fixed $j = 1, 2, \ldots, n-1$:
	\begin{align*}
		a_j^P\langle P_j, P_j \rangle + b_j^P\langle Q_j, P_j \rangle &+ c_j^P\langle R_j, P_j \rangle = - \langle x^n, P_j\rangle \\
		a_j^P\langle P_j, Q_j \rangle + b_j^P\langle Q_j, Q_j \rangle &+ c_j^P\langle R_j, Q_j \rangle = - \langle x^n, Q_j\rangle \\
		a_j^P\langle P_j, R_j \rangle + b_j^P\langle Q_j, R_j \rangle &+ c_j^P\langle R_j, R_j \rangle = - \langle x^n, R_j\rangle.
	\end{align*}
}
\frame{\frametitle{Basic Algorithm Complexity}
	$3(n-1)$ \ $3\times 3$ matrices:
	
	\vspace*{2mm}
	\ \ \ \ \ \ \ \ \ \ \ \ \ \ \ \ \ \ \ \ \ \ \ \ 
	\begin{tikzpicture}\draw[color=gray,step=.1cm] (-0.01,-0.01) grid (0.3,0.3);\end{tikzpicture}
	\begin{tikzpicture}\draw[color=gray,step=.1cm] (-0.01,-0.01) grid (0.3,0.3);\end{tikzpicture}
	\begin{tikzpicture}\draw[color=gray,step=.1cm] (-0.01,-0.01) grid (0.3,0.3);\end{tikzpicture}
	\begin{tikzpicture}\draw[color=gray,step=.1cm] (-0.01,-0.01) grid (0.3,0.3);\end{tikzpicture}
	\begin{tikzpicture}\draw[color=gray,step=.1cm] (-0.01,-0.01) grid (0.3,0.3);\end{tikzpicture}
	\begin{tikzpicture}\draw[color=gray,step=.1cm] (-0.01,-0.01) grid (0.3,0.3);\end{tikzpicture}
	\begin{tikzpicture}\draw[color=gray,step=.1cm] (-0.01,-0.01) grid (0.3,0.3);\end{tikzpicture}
	\begin{tikzpicture}\draw[color=gray,step=.1cm] (-0.01,-0.01) grid (0.3,0.3);\end{tikzpicture}
	\begin{tikzpicture}\draw[color=gray,step=.1cm] (-0.01,-0.01) grid (0.3,0.3);\end{tikzpicture}
	
	\ \ \ \ \ \ \ \ \ \ \ \ \ \ \ \ \ \ \ \ \ \ \ \ 
	\begin{tikzpicture}\draw[color=gray,step=.1cm] (-0.01,-0.01) grid (0.3,0.3);\end{tikzpicture}
	\begin{tikzpicture}\draw[color=gray,step=.1cm] (-0.01,-0.01) grid (0.3,0.3);\end{tikzpicture}
	\begin{tikzpicture}\draw[color=gray,step=.1cm] (-0.01,-0.01) grid (0.3,0.3);\end{tikzpicture}
	\begin{tikzpicture}\draw[color=gray,step=.1cm] (-0.01,-0.01) grid (0.3,0.3);\end{tikzpicture}
	\begin{tikzpicture}\draw[color=gray,step=.1cm] (-0.01,-0.01) grid (0.3,0.3);\end{tikzpicture}
	\begin{tikzpicture}\draw[color=gray,step=.1cm] (-0.01,-0.01) grid (0.3,0.3);\end{tikzpicture}
	\begin{tikzpicture}\draw[color=gray,step=.1cm] (-0.01,-0.01) grid (0.3,0.3);\end{tikzpicture}
	\begin{tikzpicture}\draw[color=gray,step=.1cm] (-0.01,-0.01) grid (0.3,0.3);\end{tikzpicture}
	\begin{tikzpicture}\draw[color=gray,step=.1cm] (-0.01,-0.01) grid (0.3,0.3);\end{tikzpicture}	
	
	\ \ \ \ \ \ \ \ \ \ \ \ \ \ \ \ \ \ \ \ \ \ \ \ 
	\begin{tikzpicture}\draw[color=gray,step=.1cm] (-0.01,-0.01) grid (0.3,0.3);\end{tikzpicture}
	\begin{tikzpicture}\draw[color=gray,step=.1cm] (-0.01,-0.01) grid (0.3,0.3);\end{tikzpicture}
	\begin{tikzpicture}\draw[color=gray,step=.1cm] (-0.01,-0.01) grid (0.3,0.3);\end{tikzpicture}
	\begin{tikzpicture}\draw[color=gray,step=.1cm] (-0.01,-0.01) grid (0.3,0.3);\end{tikzpicture}
	\begin{tikzpicture}\draw[color=gray,step=.1cm] (-0.01,-0.01) grid (0.3,0.3);\end{tikzpicture}
	\begin{tikzpicture}\draw[color=gray,step=.1cm] (-0.01,-0.01) grid (0.3,0.3);\end{tikzpicture}
	\begin{tikzpicture}\draw[color=gray,step=.1cm] (-0.01,-0.01) grid (0.3,0.3);\end{tikzpicture}
	\begin{tikzpicture}\draw[color=gray,step=.1cm] (-0.01,-0.01) grid (0.3,0.3);\end{tikzpicture}
	\begin{tikzpicture}\draw[color=gray,step=.1cm] (-0.01,-0.01) grid (0.3,0.3);\end{tikzpicture}	.
	
	\vspace*{2mm}
	Given $B_n = \{P_n, Q_n, R_n\}$ we can use Gram-Schmidt to construct $OB_n = \{\hat{P}_n, \hat{Q}_n, \hat{R}_n\}$. 
}
\frame{\frametitle{Clever Algorithm}
	Introduce recursion to similarly construct $B_n = \{P_n, Q_n, R_n\}$ with base cases $B_0 = \{1\}$ and $B_1 = \{x-\frac{1}{6}, y -\frac{1}{6}, z -\frac{1}{6}\}$:
	\begin{align*}
	P_n(x, y, z) = x&P_{n-1}(x, y, z) + \sum_{i=1}^{n-1} a_i^PP_i(x, y, z) \\
				&+ \sum_{i=1}^{n-1} b_i^PQ_i(x, y, z) + \sum_{i=1}^{n-1} c_i^PR_i(x,y,z) +d_0^P.
	\end{align*}
	Directly compute each element after coefficient determination:
	\begin{align*}
		P_n(x,y,z) = (x&-a_{n-1}^P)P_{n-1}(x,y,z) + a_{n-2}^PP_{n-2}(x,y,z) \\
				&+ b_{n-1}^PQ_{n-1}(x,y,z) + b_{n-2}^PQ_{n-2}(x,y,z) \\
				&+ c_{n-1}^PR_{n-1}(x,y,z) + c_{n-2}^PR_{n-2}(x,y,z).
	\end{align*}
}
\frame{\frametitle{Clever Algorithm Complexity}
	Six $3\times 3$ matrices: \ \ \ \ \ \begin{tikzpicture}\draw[color=gray,step=.1cm] (-0.01,-0.01) grid (0.3,0.3);\end{tikzpicture}
	\begin{tikzpicture}\draw[color=gray,step=.1cm] (-0.01,-0.01) grid (0.3,0.3);\end{tikzpicture}
	\begin{tikzpicture}\draw[color=gray,step=.1cm] (-0.01,-0.01) grid (0.3,0.3);\end{tikzpicture}
	\begin{tikzpicture}\draw[color=gray,step=.1cm] (-0.01,-0.01) grid (0.3,0.3);\end{tikzpicture}
	\begin{tikzpicture}\draw[color=gray,step=.1cm] (-0.01,-0.01) grid (0.3,0.3);\end{tikzpicture}
	\begin{tikzpicture}\draw[color=gray,step=.1cm] (-0.01,-0.01) grid (0.3,0.3);\end{tikzpicture} .
		
	\vspace*{2mm}
	Given $B_n = \{P_n, Q_n, R_n\}$ we can use Gram-Schmidt to construct $OB_n = \{\hat{P}_n, \hat{Q}_n, \hat{R}_n\}$. 
}
\frame{\frametitle{Brilliant Algorithm}
	Mutually-Orthogonal: $P_n = \langle$any linear combination of $x,y,z\rangle$.
	
	\vspace*{2mm}
	Orthonormal: mutually-orthogonal polynomials $\langle P_n, P_n \rangle = 1$.
	
	\vspace*{2mm} Again construct $OB_n = \{\hat{P}_n, \hat{Q}_n, \hat{R}_n\}$ with base cases $OB_0 = \{1\}$ and $OB_1 = \{x-\frac{1}{6}, y -\frac{1}{6}, z -\frac{1}{6}\}$:
	\begin{align*}
	P_n(x, y, z) = x&\hat{P}_{n-1}(x, y, z) + \sum_{i=1}^{n-1} a_i^P\hat{P}_i(x, y, z) \\
				&+ \sum_{i=1}^{n-1} b_i^P\hat{Q}_i(x, y, z) + \sum_{i=1}^{n-1} c_i^P\hat{R}_i(x,y,z) +d_0^P.
	\end{align*}
	Determine each element directly, but with more simplifications!
}
\frame{\frametitle{Brilliant Algorithm}
	Given mutual-orthogonality, we obtain diagonal matrices:
	\begin{align*}
		a_j^P\langle P_j, P_j \rangle +0 +0&= - \langle xP_{n-1}, P_j\rangle \\
		0+ b_j^P\langle Q_j, Q_j\rangle +0&= - \langle xP_{n-1}, Q_j\rangle \\
		0+ 0+ c_j^P\langle R_j, R_j \rangle&= - \langle xP_{n-1}, R_j\rangle.
	\end{align*}
	Given orthonormality, we obtain direct definitions:
		$$a_j^P = - \langle xP_{n-1}, P_j\rangle, \ b_j^P = - \langle xP_{n-1}, Q_j\rangle, \ c_j^P = - \langle xP_{n-1}, R_j\rangle.$$
	Thus, the computation of each element is trivial:
	\begin{align*}
		P_n(x,y,z) = (x&-a_{n-1}^P)\hat{P}_{n-1}(x,y,z) + a_{n-2}^P\hat{P}_{n-2}(x,y,z) \\
			&+ b_{n-1}^P\hat{Q}_{n-1}(x,y,z) + b_{n-2}^P\hat{Q}_{n-2}(x,y,z) \\
			&+ c_{n-1}^P\hat{R}_{n-1}(x,y,z) + c_{n-2}^P\hat{R}_{n-2}(x,y,z).
	\end{align*}
}
\frame{\frametitle{Brilliant Algorithm Complexity}
	Zero matrices.

	\vspace*{2mm}
	Gram-Schmidt immediately constructs $OB_n = \{\hat{P}_n, \hat{Q}_n, \hat{R}_n\}$.
}
\frame{\frametitle{Research Recap}
	\begin{itemize}
		\item Establish an orthogonal basis of our inner product space on $\mathbb{R} [x, y, z] / \langle xy, xz, yz \rangle$ that accurately represents the structure of [monster] polynomials on any 3D tripod.
		\item Na\"ve: arbitrary monic polynomials. \\ \ \ \ - \ Three $(3n-2) \times (3n-2)$ matrices.
		\item Basic: iterative monic polynomials. \\ \ \ \ - \ $3(n-1)$ \ $3 \times 3$ matrices.
		\item Clever: recursive monic polynomials. \\ \ \ \ - \ Six $3 \times 3$ matrices.
		\item Brilliant: orthonormal polynomials. \\ \ \ \ - \ Zero matrices.
	\end{itemize}
}





\section{}\frame{\frametitle{Questions?}}





\end{document}